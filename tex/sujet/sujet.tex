% xelatex %
%
\documentclass[10pt, a4paper ]{article}

\PassOptionsToPackage{usenames, dvipsnames, xetex}{xcolor}

\usepackage[frenchb]{babel}
\usepackage{fontspec}
% \pagestyle{empty}
\usepackage{marvosym}
\usepackage{dice}

\usepackage{tikz}
\usetikzlibrary{positioning, arrows}

% DOCUMENT LAYOUT
\usepackage{geometry}
\geometry{a4paper, textwidth=5.5in, textheight=9in, marginparsep=7pt,
marginparwidth=.6in}
\setlength\parindent{3mm}
\setlength\parskip{3mm}

% \definecolor{MidnightBlue}{rgb}{0.15,0.30,0.55}
\definecolor{BrickRed}{rgb}{0.55,0.15,0.30}

% FONTS
\usepackage{xunicode}
\usepackage{xltxtra}
\usepackage{amsmath}

% converts LaTeX specials (``quotes'' --- dashes etc.) to unicode
\defaultfontfeatures{Mapping=tex-text}

\setromanfont [Ligatures={Common}%, Numbers={OldStyle}
]{Gentium Basic}
\setsansfont [Ligatures={Common}, BoldFont={Fontin Sans Bold},
ItalicFont={Fontin Sans Italic}]{Fontin Sans}
\setmonofont[Scale=0.8]{Monaco}

% PDF SETUP
\usepackage[xetex, bookmarks, colorlinks, breaklinks, pdftitle={Rendu d'images
3D par lancer de rayons}, pdfauthor={Xavier Olive}]{hyperref}
\hypersetup{linkcolor=MidnightBlue,citecolor=BrickRed,
filecolor=black,urlcolor=MidnightBlue}

\title{Rendu d'images 3D par lancer de rayons}
\begin{document}

\noindent
\textbf{\textsf{\large IN 104 -- Rendu d'images 3D par lancer de rayons }}
\vskip1mm\hrule\url{https://github.com/xoolive/raytracer}

% \maketitle

\begin{minipage}{.9\textwidth} \sf L'objectif de ce projet est de concevoir et
    développer un projet logiciel complet tout en découvrant plusieurs méthodes
    de travail couramment utilisées dans le monde industriel. Le logiciel
    demandé sera capable de synthétiser des images par lancer de rayon.\\ Le
    langage de programmation à utiliser est le langage C. \\[5mm]
    {\color{BrickRed} Attention!} \hspace{5mm} Ce sujet est amené à être corrigé
    régulièrement. Il est recommandé de le mettre souvent à jour  depuis le
    site.\end{minipage}

% \tableofcontents

\section{Description du logiciel -- modèle physique}
\subsection{Principe général}

\paragraph{Le logiciel à écrire} est un logiciel de synthèse d'images. Il
prendra en entrée un modèle d'une scène qui est décrite par un ensemble de
facettes triangulaires.  En plus de ses propriétés géométriques, chaque facette
a des propriétés visuelles; nous nous limiterons à sa couleur et à sa
réflectivité.  En sortie, nous voulons produire l'image que pourrait voir un
observateur placé à un certain point et qui regarderait la scène modélisée dans
un champ de vision (une certaine ouverture) avec un éclairage particulier, que
nous limiterons à une seule source (ponctuelle).

Pour déterminer l'ensemble des pixels (points) qui constituent l'image, on place
sur le trajet des rayons un écran imaginaire délimité par l'ouverture sur un
plan perpendiculaire à la direction d'observation. Cette surface est ensuite
divisée en un nombre de lignes et de colonnes correspondant à la résolution
recherchée; chaque case ainsi formée étant la localisation d'un pixel. Pour les
premiers tests, une résolution faible, de l'ordre de $50\times50$ pixels, est
largement suffisante.

L'algorithme consiste à lancer un certain nombre de rayons depuis le point
d'observation en direction du modèle (figure~\ref{fig:algo}). Pour chaque pixel
de l'image, on détermine sa couleur à l'aide d'un rayon. Le rayon est lancé
depuis le point d'observation et passe par le pixel, puis va intersecter ou non
une facette du modèle.

% Il faut noter que la distance de l'écran à l'observateur n'influence pas les
% caractéristiques (nombre, origine, direction, \ldots) des rayons lancés, ce
% qui conduit à le placer à une distance arbitraire (strictement positive).

\begin{figure}[ht]
    \begin{center}
        \begin{tikzpicture}
            \path (0,0) node[anchor=east] (o) {observateur};
            \begin{scope}[scale=0.3, xshift=1.55\textwidth]
                \path (0,6) node[anchor=south] (a) {scène};
                \planepartition{{5,3,2,2},{4,2,2,1},{2,1},{1}}
            \end{scope}
            \path (1.5, 0.25) + (330:.375) node (p) {};
            \path (1.5, 0.25) + (330:.625) node (q) {};
            \path (2, 1/3) + (330:.5) node (pp) {};
            \path (pp) + (270:.15) node (qq) {};
%             \path (2, 1/3) + (330:.833) node (qq) {};
            \path (4, 2/3) + (330:1) node (ppp) {};
            \path (4, 2/3) + (330:1.67) node (qqq) {};
            \draw[BrickRed!20!white,thick] (p.center) -- (pp.center);
            \draw[BrickRed!20!white,thick] (q.center) -- (qq.center);
            \path (1.5,1.8) node[anchor=south west] (p1) {écran};
            \foreach \i in {1.5, 1, .5, 0, -.5, -1} {
                \path (1.5, \i) node (e1) {};
                \path (e1) + (330:1) node (e2) {};
                \draw (e1.center) -- (e2.center) ;
            }
            \foreach \i in {0, .25, .5, .75, 1} {
                \path (1.5, 1.5) + (330:\i) node (e1) {};
                \path (1.5, -1) + (330:\i) node (e2) {};
                \draw (e1.center) -- (e2.center) ;
            }
            \draw[BrickRed!70!white,thick] (o.east) -- (p.center);
            \draw[BrickRed!70!white,thick] (o.east) -- (q.center);
            \draw[BrickRed!70!white,thick,->,>=stealth] (pp.center) -- (ppp.center);
            \draw[BrickRed!70!white,thick,->,>=stealth] (qq.center) -- (qqq.center);
        \end{tikzpicture}
    \end{center}
    \caption{Principe général de l'algorithme}
    \label{fig:algo}
\end{figure}

\subsection{Calcul de la couleur résultante}

\paragraph{La couleur résultante} est déterminée au niveau du point
d'intersection entre le rayon et le modèle en fonction de divers paramètres:
\begin{itemize}
    \item la couleur propre de la facette, qui fait partie de la description du
        modèle;
    \item le degré d'éclairement du point d'intersection par la source lumineuse
        et la couleur de cette source;
    \item le coefficient de réflexion sur la facette intersectée,
        qui s'il est maximum (égal à 1), en fait un miroir. Avec un coefficient de
        réflexion non nul, la couleur de la lumière qui se réflechit au point
        d'intersection influence donc la couleur observée;
    \item une couleur de fond pour les rayons qui n'intersectent pas le modèle.
\end{itemize}

Les couleurs sont décrites suivant le modèle RVB (RGB en anglais): chaque
couleur est décomposée selon trois composantes (rouge, vert, bleu) et chaque
composante est  discrétisée sur une échelle de 0 (absence de cette couleur dans
le mélange) à 255 (présence maximale de cette couleur dans le mélange). Ainsi,
la couleur codée par (255,0,0) est le rouge le plus intense, (200,0,0) est un
rouge pur mais moins intense, (200,100,100) est un mélange de rouge à 50\,\%, et
de vert et de bleu à 25\,\% (le résultat est entre orange et brique). Toutes les
formules relatives aux couleurs marchent par 3, une sous-formule par composante.

La formule qui donne la couleur finale d'un pixel de l'image résultat est la
suivante:
\begin{equation}
    C = k\cdot C^R + \frac{E\cdot(1-k)}{255}\left( \overrightarrow{N}\cdot\overrightarrow{S}
    \right)\cdot C^S\cdot C^F
    \label{eqn:C}
\end{equation}

Dans cette formule:
\begin{itemize}
    \item $k$ représente le coefficient de réflexion (entre 0 et 1) de la
        facette intersectée;
    \item $C^R$ est la couleur qui se réfléchit au point d'intersection;
    \item $\overrightarrow{N}$ est la normale à la facette intersectée au point
        d'intersection;
    \item $\overrightarrow{S}$ est le vecteur de direction du point d'intersection vers la
        source lumineuse;
    \item $C^S$ est la couleur de la source lumineuse;
    \item $C^F$ est la couleur de la facette intersectée.
    \item $E$ vaut $0$ ou $1$ selon que le point d'intersection est ou non
        éclairé par la source. Il vaut 1 si la source est du bon côté de la
        facette intersectée (ce qui se traduit par un produit scalaire
        positif et s'il n'y a pas d'obstacle entre le point
        intersecté et la source lumineuse. Pour trancher ce dernier point, il
        faut savoir si une facette du modèle est dans la direction de la source
        et plus proche que cette-ci du point d'intersection; on peut le
        déterminer en lançant un rayon du point d'intersection vers la source
        lumineuse.
\end{itemize}

\subsection{Calcul de la couleur réfléchie}
\paragraph{La couleur réfléchie $C^R$} présente dans la formule~(\ref{eqn:C})
découle des lois de Snell-Descartes: le rayon réfléchi est dans le plan
d'incidence et les deux angles sont identiques
\begin{equation}
    \left( -\overrightarrow{I}, \overrightarrow{N} \right) = \left( \overrightarrow{N}, \overrightarrow{R} \right)
    \label{eqn:descartes}
\end{equation}

La figure~\ref{fig:descartes} illustre ce phénomène.


\begin{figure}[ht]
    \begin{center}
        \begin{tikzpicture}[scale=0.75]
            \path (.5,.75) node (o) {};
            \path (-1,-1) node (a) {};
            \path (0,2) node (b) {};
            \path (2,1) node (c) {};
            \draw[black!80] (a.center) -- (b.center) -- (c.center) -- (a.center);
            \draw[->,>=stealth,thick] (o.center) -- (-1,1) node[above, left]
            {\small$\overrightarrow{N}$};
            \draw[->,>=stealth,very thick,BrickRed] (-2,0) node [below left] {$\overrightarrow{I}$} -- (o.center);
            \draw[->,>=stealth,very thick,MidnightBlue] (o.center) -- (-1,2) node [above, left] {$\overrightarrow{R}$};
        \end{tikzpicture}
    \end{center}
    \caption{Loi de la réflexion de Snell-Descartes}
    \label{fig:descartes}
\end{figure}


À son tour, la couleur réfléchie se détermine en lançant un nouveau rayon, dit
\emph{secondaire}, qui peut à son tour se réfléchir. De manière à éviter tout
problème de réflexion infinie, il conviendra de limiter le nombre de réflexions
autorisées à partir d'un rayon initial.

\subsection{Format et orientation de l'écran}

On définit un repère caméra par un point observateur $O$, un point au centre de
l'écran $A$ et un point $B$ au milieu à droite sur l'écran, tel que
$\overrightarrow{AO}\, \bot\, \overrightarrow{AB}$ \footnote{Dans votre programme,
    il conviendra de vérifier que le point $B$ fourni vérifie
    $\overrightarrow{AO}\, \bot\, \overrightarrow{AB}$ et, si nécessaire, de le
    projeter orthogonalement sur le plan $\left( A, \overrightarrow{AO}
\right)$}.

Le point $C$ permettant de définir un repère sur l'écran est alors
défini tel que $\left( \overrightarrow{AB}, \overrightarrow{AC},
\overrightarrow{AO} \right)$ forme un repère orthogonal, avec $\lVert
\overrightarrow{AB} \rVert = \lVert \overrightarrow{AC} \rVert$.

On découpe alors l'écran ainsi défini en une grille en fonction de la résolution
voulue.


\begin{figure}[ht]
    \begin{center}
        \begin{tikzpicture}
            \path (0,0) node[anchor=east] (o) {O};
            \path (1.5, 0.25) + (330:.5) node (a) {};
            \path (a) + (330:.5) node (b) {};
            \path (a) + (0,1.25) node (c) {};
            \path (1.5,1.8) node[anchor=south west] (p1) {écran};
            \foreach \i in {1.5, 1, .5, 0, -.5, -1} {
                \path (1.5, \i) node (e1) {};
                \path (e1) + (330:1) node (e2) {};
                \draw[black!50] (e1.center) -- (e2.center) ;
            }
            \foreach \i in {0, .25, .5, .75, 1} {
                \path (1.5, 1.5) + (330:\i) node (e1) {};
                \path (1.5, -1) + (330:\i) node (e2) {};
                \draw[black!50] (e1.center) -- (e2.center) ;
            }
            \draw[BrickRed,ultra thick, <-, >=stealth'] (o.east) -- (a.center);
            \draw[BrickRed,ultra thick, ->, >=stealth'] (a.center) -- (b.center);
            \draw[BrickRed,ultra thick, ->, >=stealth'] (a.center) -- (c.center);
            \node[below = 1mm of a] {A};
            \node[below = 1mm of b] {B};
            \node[right = 1mm of c] {C};
        \end{tikzpicture}
    \end{center}
    \caption{Description du repère écran}
    \label{fig:repere}
\end{figure}



\subsection{Format du fichier d'entrée (.scn)}

Une scène est définie par un \og~repère caméra~\fg, une source de lumière, une
couleur de fond est un ensemble de facettes décorées. Le format des fichiers
textes que votre sous-programme de lecture devra pouvoir lire est le suivant:

\begin{itemize}
    \item toutes les lignes qui commencent par le caractère dièse \texttt{\#}
        sont des lignes de commentaires, à ignorer;
    \item une première ligne contient la position de l'observateur $x,y,z$;
    \item la deuxième ligne représente l'écran: $x_A,y_A,z_A, x_B, y_B, z_B, w,
        h$; dans l'ordre les coordonnées du point au centre de l'écran, les
        coordonnées du point à droite de l'écran, et la taille de l'image en
        sortie (largeur, hauteur).
    \item une troisième ligne contient le nombre de sommets de la scène suivie
        des coordonnées de chacun de ses points $x,y,z$, chacun sur une ligne
        différente.
    \item à la suite, une nouvelle ligne contient le nombre de facettes, puis
        pour chaque facette écrite sur une ligne différente les indices des
        sommets concernés (en commençant par $1$), le coefficient $k$ de
        réflexion de la facette, et les trois composantes RGB de sa couleur
        propre: $a,b,c,k,R,G,B$.
    \item les deux dernières lignes contiennent respectivement les trois
        composantes de la couleur de fond $R,G,B$, puis les caractéristiques de
        la source de lumière $x,y,z,R,G,B$.
\end{itemize}

Le fichier \texttt{scenes/cubes.scn} est un exemple de fichier que votre
programme doit savoir lire.

Afin de vous aider à lire le fichier, la librairie \texttt{libutil} contient une
fonction \texttt{splitLine} qui découpe une chaîne de caractères
en un tableau de mots, le découpage se faisant au niveau des espaces et des
virgules.

\subsection{Format du fichier de sortie (PPM)}

Le programme doit écrire l'image rendue au format PPM (P3). Il vous appartient
de rechercher ce qu'est un fichier PPM valide, d'en écrire un et de l'afficher
sur votre poste de travail avec le logiciel de votre choix (Sous Linux, le
logiciel \texttt{xv} est capable de lire ce type de fichiers)

Sachez qu'avant d'en arriver à l'écriture de ce fichier, vous pourrez faire des
tests de votre programme grâce à la librairie \texttt{libimgsdl}
fournie.\footnote{Des élèves en difficulté pourront faire des copies d'écran des
    rendus \texttt{libimgsdl} pour leurs compte-rendus.}

\section{Méthode de travail}

\subsection{Développement itératif}

Il est quasiment impossible de réussir à coder un programme complexe du premier
coup. Dans les cours d'initiation au langage C, on vous proposait des étapes
intermédiaires simples pour vous permettre de coder des programmes plus
complexes. Aujourd'hui, il vous appartient de définir vous-mêmes les étapes
intermédiaires: à partir d'une situation complexe, commencez par identifier des
cas particuliers simples (angles tous perpendiculaires, fichiers d'entrée à une
seule facette).

Une fois que votre programme compile et fonctionne correctement, intéressez-vous
à la généralisation. Ainsi, l'écart entre l'état courant de votre code et une
version stable est toujours faible.  Il est toujours plus facile de chercher les
erreurs dans un programme proche d'une version stable que dans un programme de
5000 lignes qui n'a jamais été compilé.

\subsection{Tests unitaires}

Lors du développement et de la maintenance d'un logiciel, il est préférable
d'écrire un jeu de tests unitaires. Il s'agit de petits programmes
indépendants\footnote{On entend par programme indépendant le fait qu'il contient
    sa propre fonction \texttt{int main()}, tout en faisant appel aux fonctions
    utilisées par le programme principal. Il convient donc de faire attention
    lors de la compilation à ce qu'une seule fonction \texttt{int main()} soit
présente dans chaque exécutable.} qui testent une par une les fonctions
implémentées dans le programme principal. En général pour chaque fonction, on
teste la correction des résultats de la fonction pour des cas simples, des cas
plus complexes et des cas aux limites du domaine.

Le programme de tests unitaires compare pour chaque test une valeur obtenue à
une valeur attendue et affiche un résumé à la fin de son exécution. L'intérêt de
tels programmes est d'assurer qu'après chaque modification, on ne régresse pas
dans les fonctionnalités fournies. Dans le monde du logiciel, on essaie de
s'interdire de partager son code source avec les autres développeurs si celui-ci
fait échouer des tests unitaires.

\subsection{\tt Makefile}

Si un programme d'un seul fichier est en général facile à compiler (la commande
\mbox{\texttt{gcc fichier.c}} suffit), les gros projets logiciels sont  plus
complexes à gérer. On compile souvent séparément chaque fichier, en faisant
appel à des fichiers d'en-tête situés dans différents répertoires, en activant
des options bien précises, et en procédant à une édition de lien avec des
librairies statiques et dynamiques à appeler dans un ordre précis.  Attendre que
tout les développeurs et utilisateurs sachent compiler toutes les parties du
projet tient de la gageure.

Afin de pallier ce problème, on utilise fréquemment l'outil \texttt{make}
qui lit un fichier nommé \mbox{\texttt{Makefile}}. Ce fichier décrit comment
compiler des sous-parties d'un projet (appelées \emph{cibles}) à partir d'un
ensemble de règles de dépendances. Pour chaque cible, le fichier donne une liste
de dépendances et une ligne de compilation. En appelant \mbox{\texttt{make
<cible>}} depuis un terminal, l'outil \texttt{make} vérifie si les
dépendances ont besoin d'être recompilées et si besoin, réexécute la commande de
compilation donnée.

Nous utiliserons dans ce projet l'outil \texttt{make}: le premier
\mbox{\texttt{Makefile}} vous permettant de terminer la partie 1 vous est
fourni. En revanche, vous devrez l'éditer pour les parties 2 et 3. Lors
de la livraison de votre code source, vous fournirez un fichier
\mbox{\texttt{Makefile}}: la commande \mbox{\texttt{make all}} devra permettre
de compiler votre projet (il vous appartient bien sûr de vérifier que la
commande fonctionne chez vous avant de soumettre votre projet); la commande
\mbox{\texttt{make run}} devra lancer votre logiciel et écrire sur le terminal
un message suffisamment explicite pour décrire ce que le logiciel devrait faire.
Enfin \mbox{\texttt{make tests}}devra compiler et exécuter tous vos tests
unitaires au même titre que le test fourni.


\subsection{Règles et bonnes pratiques (2 points)}


Pendant ce module, vous serez évalués à la fois sur vos compétences en
conception de projet, en programmation et sur votre capacité à mettre en
pratique un certain nombre de bonnes pratiques.

Pour la partie programmation, il vous est demandé de suivre les bonnes pratiques
suivantes:
\begin{itemize}
    \item les variables/fonctions commencent par une minuscule; les instructions
        du préprocesseur sont toutes en majuscules;
    \item dans chaque fichier d'en-tête (en \texttt{.h}), utilisez les
        instructions \texttt{\#ifndef}, \texttt{\#define} et \texttt{\#endif};
    \item lorsque vous faites un test d'égalité, optez pour un test avec la
        constante à gauche: préférez \texttt{if (0 == x) \{ \} } à
        \texttt{if (x == 0) \{ \}}: cette habitude permet de faire échouer la
        compilation si vous mettez un seul signe \texttt{=} au lieu de deux;
    \item compilez systématiquement avec l'option \texttt{-Wall}, et enlevez
        tous les avertissements;
    \item faites des tests unitaires pour toutes vos fonctions et relancez
        régulièrement vos tests unitaires pour vérifier que l'état de votre code
        n'a pas régressé.
\end{itemize}


\section{Objectifs}

La page \url{https://github.com/xoolive/raytracer} décrit le contenu du dossier
fourni, la manière de le télécharger et, uniquement si vous le souhaitez,
d'interagir avec git. Mais pour le moment, concentrez-vous sur la partie
\ref{sec:unit}.

La table~\ref{tab:bareme} décrit le barème de notation du module. Notez bien que
la qualité du rapport influence la note de plusieurs parties qui évaluent a
priori votre code.

Comme dans l'industrie, tout retard fera l'objet de pénalités sur la note
finale. Tout retard fera l'objet d'une pénalité de 10\,\% sur la note finale;
des retards à deux des échéances entraîneront une pénalité de 20\,\%, et ainsi
de suite.

Si vous demandez en bonne et due forme un report de délai \textbf{au moins trois
jours à l'avance}, alors, en fonction des circonstances, la pénalité pourrait
ne pas être appliquée.

Les échéances à respecter sont les suivantes:
\begin{itemize}
    \item livraison de la partie 1 le 22 avril avant 22h;
    \item livraison de la partie 2 le 17 mai avant 22h;
    \item livraison finale (sous la forme d'une archive zip, tar.gz ou d'un lien
        vers un repository github) de l'ensemble de votre espace de travail
        (codes, tests, rapport \textbf{au format PDF}); le 31 mai avant 22h.
\end{itemize}

\begin{table}
    \centering
    \begin{tabular}[ht]{|lcl|}
        \hline
        \textbf{Règles et bonnes pratiques} && 2 points\\\hline
        Code indenté, commenté; bonnes pratiques && 1 point\\
        Respect d'un développement itératif &$\star$& 1 point \\\hline\hline
        \textbf{Tests unitaires} && 3 points\\\hline
        Note rendue par le programme de test fourni && 2 points\\
        Écriture de ses propres tests & $\star$  & 1 point\\\hline\hline
        \textbf{Livraison de la fourniture} && 4 points\\\hline
        Code qui ne compile pas (tests ou exécutable) && {\color{red} -2 points}\\
        Lancer de rayon fonctionnel & $\star$ & 2 points\\
        Lecture de fichier d'entrée && 1 point\\
        Production de fichier de sortie && 1 point\\\hline\hline
        \textbf{Étude d'une problématique choisie} & $\star$ & 4 points\\\hline
        Définition du problème étudié & $\star$ & 1 point \\
        Analyse du problème & $\star$ & 3 points \\
        Bonus difficulté & $\star$ & +1 point \\\hline\hline
        \textbf{Analyse de la pratique} & $\star$ & 4 points\\\hline\hline
        \textbf{Soutenance} & & 4 points\\\hline
        Fond et forme seront notés à pondération égale &&\\\hline\hline
        \textbf{Pénalité de retard} & & 10\,\%\\\hline

    \end{tabular}

    \caption{Barème de notation. Les items marqués du symbole $\star$ doivent
    faire l'objet d'une analyse dans le rapport. Ils seront évalués au moins
partiellement, sinon entièrement,  au regard de celui-ci.}

    \label{tab:bareme}
\end{table}


\subsection{Démarrons en douceur }
\label{sec:unit}

Cette première partie du projet est fortement encadrée afin de vous permettre de
vous familiariser avec le sujet et l'environnement de travail; et d'éviter à
tous de partir sur une mauvaise piste. Les structures de données et les
objectifs sont définis.  Une simple commande à exécuter sur votre répertoire de
travail saura déterminer si vous avez rempli vos objectifs. Ainsi, les moins à
l'aise devraient pouvoir assurer leurs arrières.  Attention cependant: une
grande rigueur de travail est nécessaire; le moindre écart à la consigne fera
échouer les tests et aura un impact immédiat sur la note finale.

Un squelette minimal de code vous est fourni sur le site. Il contient des noms
de structures et des en-têtes de signatures de fonctions, à respecter.

Le fichier \texttt{tests/tests\_notes.c} contient l'ensemble des tests sur
lesquels vous serez évalués. Il fait appel aux deux fichiers d'en-tête
\texttt{geometry.h} et \texttt{intersection.h}: le fichier de geométrie contient
les descriptions des structures de données alors que le fichier
\texttt{intersection.h} contient les en-têtes des fonctions testées: ces
fonctions que vous écrirez doivent être capables de calculer l'intersection d'un
rayon avec le plan induit par une facette; et de tester l'appartenance d'un
point sur un plan à une facette.

Attention: pour une compilation complète, vous devrez ajouter sur la ligne
\textsf{TESTS\_OBJ} le nom des fichiers compilés dans lesquels vous aurez codé
toutes les fonctions nécessaires au bon fonctionnement du test. Vous pouvez
choisir librement le nom de ces fichiers pourvu qu'il y ait une cohérence.

\paragraph{Contenu de la livraison} La livraison devra être envoyée avant le
dimanche 20 avril à 22h, sous la forme d'une archive zip ou tar.gz. L'archive
sera alors décompressée pour évaluation avec la commande \texttt{make tests}.
L'archive devra contenir les dossiers \texttt{include}, \texttt{src}, et le
fichier \texttt{Makefile}. Le fichier de tests fourni ne doit pas être dans
l'archive.

\paragraph{Analyse attendue dans le rapport final} Vous devrez décrire vos
analyse et décomposition du problème, les écueils identifiés et les choix qui en
ont découlé. Décrivez les tests unitaires conçus et analysez l'apport qu'ils ont
eu dans votre projet.


\subsection{Un peu plus de liberté\ldots{} }

Vous avez réussi? Félicitations! Pour cette nouvelle étape, vous devez remplir
les objectifs présentés au début du sujet et écrire un programme de lancer de
rayons complet. Vous êtes libres dans le choix des structures de données et dans
l'écriture des tests unitaires (que vous ne devez pas vous dispenser d'écrire
pour autant!). Notez que vous devrez éditer le fichier Makefile afin que la
commande \texttt{make tests} exécute à la fois les tests notés de la première
partie et les tests que vous écrivez pour cette seconde partie.

\paragraph{Note sur l'utilisation des librairies \texttt{libutil.a} et
\texttt{libimgsdl.a}} Afin de vous faciliter la tâche, deux librairies statiques
vous sont fournies. Il s'agit de code déjà compilé et archivé dans une librairie
statique (d'où le \texttt{.a}). Le contenu des fonctions est décrit dans
\texttt{util.h} et \texttt{imgsdl.h}. \texttt{util.h} vous propose des outils
que vous n'aurez pas besoin de coder. \texttt{imgsdl.h} vous permet d'afficher
une image à l'écran sans avoir à coder l'écriture du fichier PPM.

Attention lorsque vous utiliserez \texttt{libimgsdl.a}! Cette archive contient
une fonction \texttt{main}. Lors de l'édition de lien, vous ne pourrez pas lier
à la fois \texttt{libimgsdl.a} et un objet qui a lui aussi une fonction
\texttt{main} à l'intérieur (comme par exemple \texttt{tests/tests\_notes.c}).
Vous trouverez un fichier \texttt{tests/tests\_imgsdl.c} qui vous donnera un
exemple d'utilisation à compiler et exécuter avec la commande \texttt{make
tests\_sdl}.

\paragraph{Contenu de la livraison} La livraison devra être envoyée avant le
dimanche 18 mai à 22h, sous la forme d'une archive zip, tar.gz, ou si vous êtes à
l'aise, d'un repository git\footnote{Cliquez
    \href{https://github.com/xoolive/raytracer/blob/master/github.md}{ici} et
suivez les consignes de livraison décrites.}  L'archive sera alors décompressée
pour évaluation avec la commande \texttt{make run}. Celle-ci devra, à partir du
fichier \texttt{scenes/cubes.scn},  soit générer un fichier \texttt{cubes.p3} et
afficher le chemin vers ce fichier; soit ouvrir une fenêtre (avec
\texttt{libimgsdl.a}) pour afficher le rendu.  L'archive devra contenir les
dossiers \texttt{include}, \texttt{scenes}, \texttt{src}, \texttt{tests} et le
fichier \texttt{Makefile}.


\paragraph{Analyse attendue dans le rapport final} Vous devrez décrire vos
analyse et décomposition du problème, les écueils identifiés et les choix qui en
ont découlé. Décrivez les tests unitaires conçus et analyser l'apport qu'ils ont
eu dans votre projet.


\subsection{Définissez vos objectifs! }
\label{sec:obj}

Vous pourrez attaquer cette partie quand vous aurez un logiciel fonctionnel. Une
liste d'améliorations vous sont suggérées, mais vous pouvez définir les vôtres
même si elles ne sont pas dans la liste. Pour les objectifs qui nécessitent un
développement logiciel, vous êtes entièrement libres dans l'implémentation mais
des explications sur votre approche du problème seront attendues.

Chaque objectif fera l'objet d'une présentation dans le rapport: une grande
importance sera attachée à la description du problème. Une fois le problème
posé, vous pourrez commencer votre analyse écrite, \emph{puis}, le cas échéant,
coder des améliorations\footnote{Dans ce cas, il vous sera demandé de fournir le
code qui aura servi à l'analyse lors de la livraison finale.}. Les résultats
seront présentés dans le rapport sous la forme d'une analyse écrite accompagnée
d'éléments visuels (schémas, images avant/après, courbes de performance, etc.)

Il sera préférable de se concentrer sur un seul objectif, afin de le traiter en
profondeur, plutôt que de papillonner en survolant tous les domaines.

Faites vous plaisir dans le domaine qui vous intéresse le plus!

Quelques suggestions (dans le désordre des niveaux de difficulté):
\begin{itemize}
    \item affichage d'une scène exposée à des sources de lumière multiples;
    \item transparence, réfraction (Snell-Descartes);
    \item profondeur de champ;
    \item comparaison de performances entre la fonction $1/\sqrt{x}$ et
        \texttt{fastInvSqrt} (fournie dans \texttt{libutil.a}).
    \item modélisation des couleurs en trois composantes
        teinte, saturation, valeur (HSV) plutôt que rouge, vert, bleu (RGB);
    \item parallélisation des calculs de rendu (étude papier);
    \item génération de fichiers d'entrée du logiciel à partir de scènes
        décrites en utilisant des formes géométriques courantes (sphères,
        cylindres, ellipsoïdes, théières, etc.);
    \item affichage d'une source de lumière (halo, parhélie, flare,  etc.);
    \item étude de la complexité;
    \item utilisation d'une structure d'octrees pour optimiser les calculs
        d'intersection
\end{itemize}

\subsection{Analyse de la pratique }

Le rapport final devra contenir une partie d'une ou deux pages qui analyse de
façon critique le déroulement du projet ainsi que les facteurs de réussite et
d'échecs.

\subsection{Soutenance }

Le projet se concluera par une soutenance en deux parties: 10 minutes de
présentation et 10 minutes de questions. La présentation comportera une
démonstration sur la base du code de la livraison finale, une analyse des
difficultés recontrées (y compris la problématique de la partie~\ref{sec:obj})
et des solutions mises en œuvre. Il ne sera pas nécessaire de présenter ce
qu'est le lancer de rayons.

\appendix

\section{Historique}

Le lancer de rayons est la manière historique de gérer l'imagerie 3D. C'est une
excellente manière de rendre compte des caractéristiques géométriques de la
lumière. (réflexion, réfraction, transparence)
Aujourd'hui, d'autres méthodes existent pour gérer la 3D, mais le lancer de
rayons reste néanmoins un sujet d'actualité.

L'article fondateur pour le lancé de rayon récursif a été publié en juin 1980
par T. Whitted.  Il s'agit de \emph{An improved illumination model for shaded
display}, publié dans \emph{Communications of the ACM}, (vol. 23, n°6).

\section{Lectures recommandées}

Pour ceux qui veulent en savoir plus, voici quelques références digne d'intérêt
sur le site de Wikipedia (en anglais):
\begin{itemize}
    \item l'article sur le lancer de rayons:\\
        \url{https://en.wikipedia.org/wiki/Ray\_tracing\_\%28graphics\%29}
    \item l'article sur l'infographie:\\
        \url{https://en.wikipedia.org/wiki/Computer\_graphics}
    \item l'article sur le langage de programmation Cg (C for graphics),
        développé par NVIDIA pour faire calculer un rendu d'images par la carte
        graphique (GPU) au lieu du CPU traditionnel:\\
        \url{https://en.wikipedia.org/wiki/Cg\_\%28programming\_language\%29}
    \item l'article sur le format PPM:\\
        \url{https://en.wikipedia.org/wiki/Netpbm_format}
\end{itemize}




\end{document}
